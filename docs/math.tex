\documentclass[12pt]{article}
\usepackage[T1]{fontenc}
\usepackage[width=16cm, height=21cm]{geometry}
\usepackage[ngerman]{babel}
\usepackage[hidelinks]{hyperref}

\usepackage{amsmath}
\usepackage{bm}

\usepackage{tabularx}
\usepackage{booktabs}

\usepackage{graphicx}
\usepackage{subcaption}

\usepackage{csquotes}
\usepackage{enumitem}
\usepackage{textcomp}
\usepackage{pgfplots}
\pgfplotsset{compat=1.17}

\usepackage{listings, listings-rust}

\usepackage{fontspec}
\defaultfontfeatures{Mapping=tex-text}
\defaultfontfeatures{Ligatures=TeX}
\setlength{\parindent}{0em}

\newcommand*{\tageq}{\refstepcounter{equation}\tag{\theequation}}

\newcommand{\abs}[1]{\left|#1\right|}
\newcommand{\at}[2]{\left.#1\right|_{#2}}
\newcommand{\deriv}[2]{\frac{\mathrm{d}#1}{\mathrm{d#2}}}

\begin{document}
\author{Hawo Höfer}
\title{Optical Raytracing Demo in Rust}
\maketitle
\tableofcontents
\clearpage

This raytracer is a 2D raytracer, with (for now) rather limited capabilities. It is a
Demo and should not be used for any serious applications.

\clearpage
\section{Basic Principles}
We represent a ray by a pair of coordinates in the x-y-plane, the angle it propagates at,
and the current refractive index of the material it propagates in. We apply Snell's law
of refraction to compute refraction at boundaries:
$$n_1 \sin(\alpha) = n_2 \sin(\beta),$$
where $n_i$ are the refractive indices of the materials before and after the boundary,
and $\alpha, \beta$ are the incident / refracted angles respectively.
% TODO: add illustration for Refraction


For computing the incident angle, we need to specify a way to represent boundaries. We
can then test them for intersection. This raytracer works by using an iterative process
to compute new ray directions and starting points. The top-level algorithm is:
\begin{enumerate}
  \item Compute the next intersection location. (If no intersection can be found, discard
    this sample point.)
  \item Compute the incident angle.
  \item Apply Snell's law to compute the refracted angle.
  \item Move to step 1.
\end{enumerate}

\clearpage
\section{Computing Intersections and Intersection Angles}
Depending on the type of boundary, computing the intersection can become increasingly
difficult. Currently, this raytracer implements three Boundary types:
\begin{enumerate}
  \item vertical lines,
  \item circular boundaries, and
  \item conic boundaries.
\end{enumerate}

We use rust Enums to represent the boundaries:
\begin{lstlisting}[caption={Representation of Boundary Types\label{lst:boundary_types}},language=Rust]
  enum BoundaryType {
      Line {
          opt_idx: f64,
          midpoint: f64,
          radius: f64,
      },
      Spherical {
          opt_idx: f64,
          midpoint: f64,
          radius: f64,
          height: f64,
      },
      Conic {
          opt_idx: f64,
          midpoint: f64,
          radius: f64,
          conic_param: f64,
          height: f64,
      },
  }
\end{lstlisting}

Each of those boundary types has their own way of computing a point of intersection.

\clearpage
\subsection{Vertical Lines}
Vertical line boundaries are represented by a coordinate on the optical axis, and a
radius, indicating the maximum distance from the optical axis (see Listing 
\ref{lst:boundary_types}).

Intersecting vertical lines is rather simple. From the current ray starting point, we
have
\begin{itemize}
  \item the distance $\Delta x$ to the boundary along the optical axis (in $x$), and
  \item the distance $\Delta y$ in $y$ the ray travels, depending on its angle.
\end{itemize}
The point of intersection will be at the midpoint coordinate of the Boundary
(x-coordinate along the optical axis). The y-coordinate will depend on the angle of the
ray:
$$y_\text{inter} = y_0 + \Delta x \cdot \tan{\varphi},$$
where $y_0$ and $\varphi$ are the starting coordinate of the ray and its angle,
respectively. 

We test for intersection by comparing absolute values of $y_\text{inter}$ and the
boundaries radius.

The intersection angle will be the angle of the ray. This can be seen easily from the
illustration.
% todo: add illustration for vertical line intersection


\clearpage
\subsection{Spherical Boundaries}
Compared to vertical lines, spherical boundaries are a bit more tricky. We represent them
using a midpoint on the optical axis, the circular radius, and a height, which is similar
to the radius parameter of vertical line boundaries (see Listing
\ref{lst:boundary_types}).

\paragraph{Intersection} For intersection, we solve the following system of equations:
\begin{align}
  \label{eq:sph_0} (x_\text{inter} =)\quad x_0 + \Delta x & = m - r \cos\varphi_\text{c}\\
  \label{eq:sph_1} (y_\text{inter} =)\quad y_0 + \Delta y & = r \sin\varphi_\text{c},
\end{align}
where $\Delta y = \Delta x \tan{\varphi}$, $\varphi_\text{c}$ is the angle of
intersection on the circular boundary, and $m$ is the midpoint coordinate of the boundary
on the optical axis.

We use equation \eqref{eq:sph_1} to obtain
\begin{equation}\label{eq:sph_dx}
  \Delta x = \frac{r \sin\varphi_\text{c} - y_0}{\tan\varphi}.
\end{equation}
In equation \eqref{eq:sph_0}:
\begin{align*}
  x_0 + \frac{r \sin\varphi_\text{c} - y_0}{\tan\varphi}
    &= m - r \cos\varphi_\text{c}\\
  x_0 \tan\varphi + r \sin\varphi_\text{c} - y_0 
    &= m \tan\varphi - r \tan\varphi \cos\varphi_\text{c}\\
%
  \frac{\tan(\varphi)(m - x_0) + y_0}{r}
    & = \sin\varphi_\text{c} + \tan\varphi\cos\varphi_\text{c}\\
    & = \sqrt{1 + \tan^2\varphi} \sin(\varphi_\text{c} + \arctan(\tan(\varphi)))\\
    & = \sqrt{\frac{1}{\cos^2\varphi}} \sin(\varphi_\text{c} + \varphi)\\
    & = \frac{\sin(\varphi_\text{c} + \varphi)}{\abs{\cos\varphi}}\\
%
  \sin(\varphi_\text{c} + \varphi)
    & = \frac{\abs{\cos\varphi}}{r}(\tan(\varphi) (m - x_0) + y_0)\\
  \tageq \label{eq:sph_phic} \varphi_\text{c} 
    &= \arcsin\left(\frac{\abs{\cos\varphi}}{r}(\tan(\varphi) (m - x_0) + y_0)\right)
\end{align*}
We can then compute the $x$-coordinate of the intersection by using the results from
equation \eqref{eq:sph_phic} in equations \eqref{eq:sph_0} and \eqref{eq:sph_1}.

Note that equation \eqref{eq:sph_phic} will not yield an answer for $\varphi_\text{c}$ if
$\abs{\frac{\abs{\cos\varphi}}{r} (\tan(\varphi)(m - x_0) + y_0)} > 1.0$. This this will
be our necessary criterion for intersection. The sufficient criterion is
$$\abs{y_\text{inter}} \leq y_\text{max},$$
where $y_\text{max}$ is the maximum distance of the lens from the optical axis.

\paragraph{Angle of Incidence} Using geometric reasoning, we find the tangent angle at
the intersection point
$$\varphi_\text{tangent} = \frac{\pi}{2} - \varphi_\text{c},$$
and the angle of incidence
$$\alpha = \varphi + \varphi_\text{c}.$$
% TODO: Add illustration for circular intersection
\begin{figure}[h]
  \centering
  \begin{tikzpicture}
    \begin{axis}[
      width = 0.7\textwidth,
      height = 0.7\textwidth,
      samples = 100,
      xlabel = {x},
      ylabel = {y},
      x label style = {at = {(axis description cs:1.0,0)}},
      y label style = {at = {(axis description cs:0,1.0)}, rotate=-90},
      line width = 1pt,
      axis y line = left,
      axis x line = bottom,
      axis line style = {->},
      legend cell align = left,
      unit vector ratio* = 1 1 1,
    ]
    % circle
    \addplot[color=black, domain=2:2.5]{sqrt(2^2 - (x - 4)^2)};
    \addplot[color=black, domain=2:2.5]{-sqrt(2^2 - (x - 4)^2)};

    % ray
    \addplot[color=green, domain=0:(4 - sqrt(3))]{x / (4 - sqrt(3))};
    % tangent
    \addplot[color=red, domain=(4 - sqrt(3) - 1 / sqrt(3)):2.5]{(x - (4 - sqrt(3))) * sqrt(3) + 1};
    % normal
    \addplot[color=black, domain=(4 - sqrt(3) - 1 / sqrt(3)):4 - sqrt(3) + 1 / sqrt(3)]
      {(x - (4 - sqrt(3))) * tan(atan(sqrt(3)) - 90) + 1};
    \addplot[color=black, dashed, domain=0:4 - sqrt(3) + 1]{0};
    % refracted ray
    \addplot[color=green, domain=(4 - sqrt(3)): 3] {(x - 4 + sqrt(3)) * tan(6.3144) + 1};


    \end{axis}
  \end{tikzpicture}
\end{figure}
\clearpage
\subsection{Conic Boundaries}
Conic lines can be parametrised by the following equation:
\begin{equation}\label{eq:conic_parametr}
  x(y) = \frac{y^2}{R \left(1 + \sqrt{1 - (1 + \kappa) \frac{y^2}{R^2}}\right)}
\end{equation}
With conic parameter $\kappa$ and radius $R$.

$\kappa$ is used to determine the form of profile denoted by $x(y)$.
\begin{itemize}
  \item $\kappa < -1$ yields a hyperbola
  \item $\kappa = -1$ yields a parabola
  \item $\kappa \in (-1, 0)$ yields an ellipse (prolate spheroid)
  \item $\kappa = 0$ yields a sphere
  \item $\kappa > 0$ yields an ellipse (oblate spheroid)
\end{itemize}

\paragraph{Intersection} To compute the intersection, we solve the system of equations
\begin{align}
  \label{eq:conic_0} x_0 + \Delta x & = m + x(y_\text{inter})\\
  \label{eq:conic_1} y_0 + \Delta x \tan\varphi & = y_\text{inter}
\end{align}
Solving equation \eqref{eq:conic_0} for $y_\text{inter}$ yields

\begin{equation}\label{eq:conic_y_inter}
  y_\text{inter} = z^{-1} (x_0 + \Delta x - m),
\end{equation}
where $(\cdot)^{-1}$ denotes the inverse function of $(\cdot)$.

We compute the inverse of $x(y)$ (see equation \eqref{eq:conic_parametr}). For brevity,
we write $x$ instead of $x(y)$.
\begin{align*}
  x & = \frac{y^2}{R \left(1 + \sqrt{1 - (1 + \kappa)\frac{y^2}{R^2}}\right)}\\
  x R \left(1 + \sqrt{1 - (1 + \kappa)\frac{y^2}{R^2}}\right) & = y^2\\
  x R + x \sqrt{R^2 - y^2(1 + \kappa)} & = y^2 \\
  \left(R x - y^2\right)^2 & = x^2 R^2 - x^2 y^2 (1 + \kappa)\\
  x^2 R^2 - 2 x R y^2 + y^4 & = x^2 R^2 - x^2 y^2 (1 + \kappa)\\
  y^4 + y^2 \left(x^2 (1 + \kappa) - 2 x R\right) &= 0\\
\end{align*}
Let $\rho = y^2$. Then
$$
  \rho_{1, 2} = - \frac{x^2 (1 + \kappa) - 2 x R}{2}
    \pm \abs{\frac{x^2 (1 + \kappa) - 2 x R}{2}}.
$$

Therefore (we discard the solution $\rho = 0$)
\begin{equation}\label{eq:conic_inv}
  y(x) = y^{-1}(x) = \pm \sqrt{2 x R - x^2(1 + \kappa)}
\end{equation}

We can now solve equations $\eqref{eq:conic_y_inter}$ and \eqref{eq:conic_1}.

\begin{align*}
  y_\text{inter} = y_0 + \Delta x \tan{\varphi} 
    & = \pm \sqrt{2R * (x_0 + \Delta x - m) - (x_0 + \Delta x - m)^2 (1 + \kappa)}\\
  y_0^2 + 2 \Delta x y_0 \tan\varphi + \Delta x^2 \tan^2\varphi
    & = 2 R (x_0 - m) + 2 R \Delta x \\ & \phantom{==} 
        - \left((x_0 - m)^2 + 2 \Delta x (x_0 - m) + \Delta x^2\right)(1 + \kappa)\\
  0 & = \underbrace{\-\left(1 + \kappa +\tan^2\varphi\right)}_{:= c_0}
        \Delta x^2 \\ & \phantom{==}
        + \underbrace{2 \left(R - (1 + \kappa)(x_0 - m) - y_0 \tan\varphi\right)}_{:=c_1} 
        \Delta x\\ &\phantom{==}
        + \underbrace{2 \left(R (x_0 - m) - (1 + \kappa)(x_0 - m)^2 - y_0^2\right)}_{:=c_2}
\end{align*}

We now have
\begin{equation}\label{eq:conic_dx}
  \Delta x_{1, 2} = - \frac{c_1}{2 c_0} 
    \pm \sqrt{\left(\frac{c_1}{2 c_0}\right)^2 + \frac{c_2}{c_0}},
\end{equation}
and can use this result to compute the intersection at $x_\text{inter}$. We have two
solutions,and need to decide which one to use.
We select a solution based on the sign of the Radius. If the radius is positive, the lens
will be convex, and we want to select the smaller $\Delta x$. If the radius is negative,
we want to select the larger $\Delta x$.
Now, we can compute $y_\text{inter}$ and $x_\text{inter}$.

\paragraph{Angle of Incidence}
The tangent angle at the intersection can be obtained by using the derivative of the
inverse conic function given in equation \eqref{eq:conic_inv}.
\begin{equation}\label{eq:conic_inv_deriv}
  \deriv{y}{x} = \pm \frac{R - (1 + \kappa) x)}{\sqrt{2 x R - x^2(1 + \kappa)}}
\end{equation}
We chose, which derivative we want to use based on the sign of $y_\text{inter}$.
The angle of the boundary is given by
\begin{equation}\label{eq:conic_inter_angle}
  \varphi_\text{inter} = \arctan\left(\at{\deriv{y}{x}}{x = x_\text{inter}}\right)
\end{equation}

\end{document}
